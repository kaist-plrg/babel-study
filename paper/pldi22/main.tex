%% For double-blind review submission, w/o CCS and ACM Reference (max submission space)
\documentclass[sigplan,10pt,review,anonymous]{acmart}
\settopmatter{printfolios=true,printccs=false,printacmref=false}
%% For double-blind review submission, w/ CCS and ACM Reference
%\documentclass[sigplan,review,anonymous]{acmart}\settopmatter{printfolios=true}
%% For single-blind review submission, w/o CCS and ACM Reference (max submission space)
%\documentclass[sigplan,review]{acmart}\settopmatter{printfolios=true,printccs=false,printacmref=false}
%% For single-blind review submission, w/ CCS and ACM Reference
%\documentclass[sigplan,review]{acmart}\settopmatter{printfolios=true}
%% For final camera-ready submission, w/ required CCS and ACM Reference
%\documentclass[sigplan]{acmart}\settopmatter{}


%% Conference information
%% Supplied to authors by publisher for camera-ready submission;
%% use defaults for review submission.
\acmConference[PL'18]{ACM SIGPLAN Conference on Programming Languages}{January 01--03, 2018}{New York, NY, USA}
\acmYear{2018}
\acmISBN{} % \acmISBN{978-x-xxxx-xxxx-x/YY/MM}
\acmDOI{} % \acmDOI{10.1145/nnnnnnn.nnnnnnn}
\startPage{1}

%% Copyright information
%% Supplied to authors (based on authors' rights management selection;
%% see authors.acm.org) by publisher for camera-ready submission;
%% use 'none' for review submission.
\setcopyright{none}
%\setcopyright{acmcopyright}
%\setcopyright{acmlicensed}
%\setcopyright{rightsretained}
%\copyrightyear{2018}           %% If different from \acmYear

%% Bibliography style
\bibliographystyle{ACM-Reference-Format}
%% Citation style
%\citestyle{acmauthoryear}  %% For author/year citations
%\citestyle{acmnumeric}     %% For numeric citations
%\setcitestyle{nosort}      %% With 'acmnumeric', to disable automatic
                            %% sorting of references within a single citation;
                            %% e.g., \cite{Smith99,Carpenter05,Baker12}
                            %% rendered as [14,5,2] rather than [2,5,14].
%\setcitesyle{nocompress}   %% With 'acmnumeric', to disable automatic
                            %% compression of sequential references within a
                            %% single citation;
                            %% e.g., \cite{Baker12,Baker14,Baker16}
                            %% rendered as [2,3,4] rather than [2-4].


%%%%%%%%%%%%%%%%%%%%%%%%%%%%%%%%%%%%%%%%%%%%%%%%%%%%%%%%%%%%%%%%%%%%%%
%% Note: Authors migrating a paper from traditional SIGPLAN
%% proceedings format to PACMPL format must update the
%% '\documentclass' and topmatter commands above; see
%% 'acmart-pacmpl-template.tex'.
%%%%%%%%%%%%%%%%%%%%%%%%%%%%%%%%%%%%%%%%%%%%%%%%%%%%%%%%%%%%%%%%%%%%%%

%% macros with packages
% load packages
\usepackage{float}
\usepackage{algorithmic}
\usepackage{amsmath,amsfonts}
\usepackage[ruled, vlined]{algorithm2e}
\usepackage{graphicx}
\usepackage{textcomp}
\usepackage{xcolor}
\usepackage{soul}
\usepackage{listings}
\usepackage{caption}
\usepackage{subcaption}
\usepackage{multirow}
\usepackage{booktabs}
\usepackage{makecell}
\usepackage{galois}
\usepackage{mathpartir}
\usepackage{bussproofs}
\usepackage{mathtools}
\usepackage{colortbl}
\usepackage{hhline}
\usepackage{stmaryrd}
\usepackage{microtype}
\usepackage{hyperref}
\usepackage{balance}
\usepackage{adjustbox}
\usepackage{tikz}
\usepackage{csquotes}

% box
\newcommand{\cfbox}[2]{%
  \colorlet{currentcolor}{.}%
  {\color{#1}%
  \fbox{\color{currentcolor}#2}}%
}
\newcommand{\lcolorbox}[2]{\adjustbox{padding=0ex 1ex 1ex 1ex, bgcolor=#1}{#2}}
\newcommand{\rcolorbox}[2]{\adjustbox{padding=1ex 1ex 0ex 1ex, bgcolor=#1}{#2}}

% table rules
\newcolumntype{?}{!{\vrule width 1pt}}
\newcommand*{\belowrulesepcolor}[1]{%
  \noalign{%
    \kern-\belowrulesep
    \begingroup
      \color{#1}%
      \hrule height\belowrulesep
    \endgroup
  }%
}
\newcommand*{\aboverulesepcolor}[1]{%
  \noalign{%
    \begingroup
      \color{#1}%
      \hrule height\aboverulesep
    \endgroup
    \kern-\aboverulesep
  }%
}

% colors
\definecolor{gainsboro}{rgb}{0.86, 0.86, 0.86}
\definecolor{dkgreen}{rgb}{0, 0.5, 0}
\definecolor{lightred}{rgb}{0.93, 0.57 0.52}
\definecolor{esnt}{rgb}{0.20, 0.20, 0.20}
\definecolor{esparam}{rgb}{0.16, 0.63, 0.59}
\definecolor{esalg}{rgb}{0.12, 0.42, 0.65}
\definecolor{esvar}{rgb}{0.16, 0.63, 0.59}
\definecolor{gray1}{rgb}{0.95, 0.95, 0.95}
\definecolor{gray2}{rgb}{0.85, 0.85, 0.85}
\definecolor{gray3}{rgb}{0.75, 0.75, 0.75}

% basic
\newcommand{\inblue}[1]{{\color{blue}{#1}}}
\newcommand{\inred}[1]{{\color{red}{#1}}}
\newcommand{\x}[1]{\inred{#1}}
\newcommand{\y}[1]{\textbf{\inred{#1}}}
\newcommand{\todo}{\inred{TODO}}
\newcommand{\powerset}{\mathcal{P}}
\newcommand{\tif}{\text{if} \; }
\newcommand{\telse}{\text{otherwise}}
\newcommand{\tst}{{\; \text{s.t.} \; }}

% tool name
\newcommand{\name}[1]{\textsf{#1}}
\newcommand{\sname}[1]{\name{\small #1}}
\newcommand{\stextbf}[1]{\textbf{\small #1}}
\newcommand{\jiset}{\sname{JISET}}
\newcommand{\ires}{\sname{IR}_\sname{ES}}
\newcommand{\jest}{\sname{JEST}}
\newcommand{\esmeta}{\sname{ESMeta}}
\newcommand{\jstar}{\sname{JSTAR}}
\newcommand{\jsaver}{\sname{JSAVER}}
\newcommand{\lambdajs}{\lambda_\text{JS}}
\newcommand{\jscert}{\text{JSCert}}
\newcommand{\jsref}{\text{JSRef}}
\newcommand{\kjava}{\text{K-Java}}
\newcommand{\kjs}{\text{KJS}}
\newcommand{\javert}{\text{JaVerT}}
\newcommand{\jsil}{\text{JSIL}}
\newcommand{\comfort}{\textsc{Comfort}}
\newcommand{\kframework}{\mathbb{K}}

% our tool name
% TODO change name? which one?
\newcommand{\tool}{\jest_{\sname{fs}}}

% JavaScript code style
\lstdefinelanguage{JavaScript}{
  keywords={async, await, break, case, catch, class, const, continue, debugger,
    default, delete, do, else, enum, export, extends, false, finally, for,
    function, if, import, in, of, instanceof, new, null, return, super, switch,
    this, throw, true, try, typeof, let, var, void, while, with, yield},
  keywordstyle=\color{blue}\bfseries,
  ndkeywordstyle=\color{darkgray}\bfseries,
  identifierstyle=\color{black},
  numberstyle=\tiny\color{darkgray},
  numbers=none,
  numbersep=5pt,
  sensitive=false,
  comment=[l]{//},
  morecomment=[s]{/*}{*/},
  commentstyle=\color{dkgreen},
  stringstyle=\color{red}\ttfamily,
  morestring=[b]',
  morestring=[b]",
  morestring=[b]`
}
\lstdefinestyle{JS}{
  language=JavaScript,
  extendedchars=true,
  basicstyle=\small\ttfamily,
  showstringspaces=false,
  showspaces=false,
  tabsize=2,
  breaklines=true,
  showtabs=false,
  captionpos=b
}

% codes
\newcommand{\jscode}[1]{\text{\lstinline[style=JS]!#1!}}
\newcommand{\scode}[1]{\texttt{\small{#1}}}

% ECMA-262
\newcommand{\esnt}[1]{\textit{\color{esnt}#1}}
\newcommand{\esparam}[1]{\text{\color{esparam}#1}}
\newcommand{\esntp}[2]{\esnt{#1}_\esparam{[#2]}}
\newcommand{\est}[1]{\textbf{\texttt{#1}}}
\newcommand{\esalg}[1]{{\color{esalg}#1}}
\newcommand{\esvar}[1]{\textit{\color{esvar}#1}}
\newcommand{\esval}[1]{\textbf{#1}}
\newcommand{\escode}[1]{\textbf{\texttt{#1}}}
\newcommand{\esconst}[1]{\name{#1}}
\newcommand{\lab}[1]{{}^{\textbf{#1}}}

% graph coverages
\newcommand{\graph}{\mathbb{G}}
\newcommand{\nodeset}{\mathbb{N}}
\newcommand{\node}{n}
\newcommand{\nodes}{\overline{\node}}
\newcommand{\inodeset}{\nodeset_\iota}
\newcommand{\fnodeset}{\nodeset_f}
\newcommand{\edgeset}{\mathbb{E}}
\newcommand{\edge}[1]{\xrightarrow{#1}}
\newcommand{\call}{\edge{\name{call}}}
\newcommand{\ret}{\edge{\name{ret}}}
\newcommand{\tedge}{\edge{\name{\#t}}}
\newcommand{\fedge}{\edge{\name{\#f}}}
\newcommand{\annotset}{\mathbb{A}}
\newcommand{\annot}{a}
\newcommand{\patset}[1]{\mathbb{P}_{#1}}
\newcommand{\pat}{p}
\newcommand{\addpat}{\pat_\name{add}}
\newcommand{\subpat}{\pat_\name{sub}}
\newcommand{\patmap}[1]{\name{path}_{#1}}
\newcommand{\getfirst}{\name{first}}
\newcommand{\getlast}{\name{last}}
\newcommand{\testset}{\mathbb{T}}
\newcommand{\test}{t}
\newcommand{\addtest}{\test_\name{add}}
\newcommand{\subtest}{\test_\name{sub}}
\newcommand{\prefix}{\preceq}
\newcommand{\subpath}{\sqsubseteq}

% graph coverage
\newcommand{\cover}{\overset{\name{cover}}{\sim}}
\newcommand{\trset}[1]{\mathbb{R}_{#1}}
\newcommand{\tr}{r}
\newcommand{\cov}[1]{C_{#1}}
\newcommand{\nodecov}[1]{\cov{#1}^{\name{node}}}
\newcommand{\kpathcov}[2]{\cov{#2}^{{#1}\name{-path}}}
\newcommand{\norm}[1]{\lVert{#1}\rVert}
\newcommand{\sat}{\vdash}

% feature-sensitive (FS) coverage
\newcommand{\featset}{\mathbb{F}}
\newcommand{\feat}{f}
\newcommand{\addfeat}{\feat_{\name{add}}}
\newcommand{\subfeat}{\feat_{\name{sub}}}
\newcommand{\idfeat}{\feat_{\name{id}}}
\newcommand{\numfeat}{\feat_{\name{B:Number}}}
\newcommand{\feats}{\overline{\feat}}
\newcommand{\featmap}{\name{feat}}
\newcommand{\extfeat}{\name{ext}_\featset}
\newcommand{\extfeats}[1]{\extfeat^{#1}}
\newcommand{\css}[1]{{#1}\!\mid_{\name{call}}}
\newcommand{\fcov}[1]{\cov{#1}^{\name{FS}}}
\newcommand{\fnodecov}[1]{\cov{#1}^{\name{FS}[\name{node}]}}
\newcommand{\ftrset}[1]{\trset{#1}^{\name{FS}}}
\newcommand{\kfcov}[2]{\cov{#2}^{{#1}\name{-FS}}}
\newcommand{\kfnodecov}[2]{\cov{#2}^{{#1}\name{-FS}[\name{node}]}}
\newcommand{\kftrset}[2]{\trset{#2}^{{#1}\name{-FS}}}
\newcommand{\subs}{\rhd}

% feature-call-path-sensitive (FCPS) coverage
\newcommand{\fcpset}{\featset_\name{cp}}
\newcommand{\fcp}{\feat_\name{cp}}
\newcommand{\fcps}{\overline{\fcp}}
\newcommand{\extfcp}{\name{ext}_{\fcpset}}
\newcommand{\extfcps}[1]{\extfcp^{#1}}
\newcommand{\fcpcov}[1]{\cov{#1}^{\name{FCPS}}}
\newcommand{\fcpnodecov}[1]{\cov{#1}^{\name{FCPS}[\name{node}]}}
\newcommand{\fcptrset}[1]{\trset{#1}^{\name{FCPS}}}
\newcommand{\kfcpcov}[2]{\cov{#2}^{{#1}\name{-FCPS}}}
\newcommand{\kfcpnodecov}[2]{\cov{#2}^{{#1}\name{-FCPS}[\name{node}]}}
\newcommand{\kfcptrset}[2]{\trset{#2}^{{#1}\name{-FCPS}}}

% venn diagram
\tikzset{filled/.style={fill=gray, draw=none}}
\newcommand{\venn}[6]{%
  \begin{tikzpicture}
    \def\radius{8.7mm}
    \def\lcircle{(l) circle (\radius)}
    \def\rcircle{(r) circle (\radius)}
    \coordinate (l);
    \coordinate[xshift=\radius] (r);
    \begin{scope}
      \clip \lcircle;
      \draw[filled, opacity=#4, even odd rule] \lcircle \rcircle;
    \end{scope}
    \begin{scope}
      \clip \lcircle;
      \fill[filled, opacity=#5] {\rcircle};
    \end{scope}
    \begin{scope}
      \clip \rcircle;
      \draw[filled, opacity=#6, even odd rule] \lcircle \rcircle;
    \end{scope}
    \draw \lcircle;
    \draw \rcircle;
    \node[xshift=-4.4mm, rotate=45] at (l) {\footnotesize #1};
    \node[xshift=+4.4mm, rotate=45] at (l) {\footnotesize #2};
    \node[xshift=+4.4mm, rotate=45] at (r) {\footnotesize #3};
    \node[xshift=-7pt,yshift=\radius+10pt] at (l) {\small Test262};
    \node[xshift=+7pt,yshift=\radius+9pt] at (r) {$\tool$};
  \end{tikzpicture}
  \vspace*{-.3em}
}


%% start document
\begin{document}

%% Title information
\title[An Empirical Study of Desugaring in JavaScript]
{An Empirical Study of Desugaring in JavaScript}

%% Author information
%% Contents and number of authors suppressed with 'anonymous'.
%% Each author should be introduced by \author, followed by
%% \authornote (optional), \orcid (optional), \affiliation, and
%% \email.
%% An author may have multiple affiliations and/or emails; repeat the
%% appropriate command.
%% Many elements are not rendered, but should be provided for metadata
%% extraction tools.
\author{Jihyeok Park}
\orcid{0000-0001-8387-1984}
\affiliation{
  \department{School of Computing}
  \institution{KAIST}
  \city{Daejeon}
  \country{South Korea}
}
\email{jhpark0223@kaist.ac.kr}

\author{Dongjun Youn}
\orcid{0000-0002-5766-2035}
\affiliation{
  \department{School of Computing}
  \institution{KAIST}
  \city{Daejeon}
  \country{South Korea}
}
\email{f52985@kaist.ac.kr}

\author{Hyerin Park}
\orcid{\todo}
\affiliation{
  \department{School of Computing}
  \institution{KAIST}
  \city{Daejeon}
  \country{South Korea}
}
\email{hyerin.park@kaist.ac.kr}

\author{Sukyoung Ryu}
\orcid{0000-0002-0019-9772}
\affiliation{
  \department{School of Computing}
  \institution{KAIST}
  \city{Daejeon}
  \country{South Korea}
}
\email{sryu.cs@kaist.ac.kr}

%% Abstract
%% Note: \begin{abstract}...\end{abstract} environment must come
%% before \maketitle command
\begin{abstract}
  A \textit{conformance test} suite is essential to support consistent execution
  environments for JavaScript.
  %
  It is challenging to automatically synthesize them because of 1) the complex
  language semantics with a highly dynamic nature and 2) the fast-evolving
  language specification written in a natural language, English.
  %
  Researchers have presented a way to automatically synthesize conformance tests
  for JavaScript using \textit{coverage-guided fuzzing} with the control-flow
  graph (CFG) in the language specification.
  %
  However, existing work utilizes simple \textit{node/branch coverages}.
  %
  We found that their test requirements are not enough to distinguish the
  semantics of different \textit{language features} defined with shared abstract
  algorithms in the specification.
  %
  As a result, it weakens the fuzzing guidance and even removes meaningful
  conformance tests in the final test suite.
  %
  In addition, existing work focuses on conformance checks only for engines but
  does not consider transpilers even though developers heavily utilize them.

  %----------------------------------------%

  This paper introduces a novel \textit{feature-sensitive coverage}, which
  discriminates the test requirements with their enclosing language features.
  %
  We observe that specific abstract algorithms in the JavaScript language
  specification are coupled with language features.
  %
  We utilize this information to enhance the quality of the coverage-guided
  fuzzing for conformance test synthesis.
  %
  fuzzing for conformance test synthesis.
  %
  We implemented $\tool$ by extending a state-of-the-art JavaScript conformance
  test synthesizer, $\jest$, with our feature-sensitive coverage.
  %
  For the latest language specification (ES13, 2022), our tool automatically
  synthesized \inred{5,000} conformance tests in \inred{100} hours.
  %
  We checked conformance of not only engines but also transpilers with the
  synthesized conformance tests for evaluation.
  %
  The evaluation targets were \inred{eight} mainstream tools (\inred{four}
  engines and \inred{four} transpilers), and we discovered bugs in all of them.
  %
  Our tool detected \inred{50} unique conformance bugs (\inred{20} in engines
  and \inred{30} in transpilers), while the baseline tool detected only
  \inred{16} engine bugs.
  %
  We had reported all detected bugs, developers confirmed all of them, and
  \inred{40} were newly discovered bugs.
\end{abstract}


% TODO in camera-ready
%% 2012 ACM Computing Classification System (CSS) concepts
%% Generate at 'http://dl.acm.org/ccs/ccs.cfm'.
% \begin{CCSXML}
% <ccs2012>
% <concept>
% <concept_id>10011007.10011006.10011008</concept_id>
% <concept_desc>Software and its engineering~General programming languages</concept_desc>
% <concept_significance>500</concept_significance>
% </concept>
% <concept>
% <concept_id>10003456.10003457.10003521.10003525</concept_id>
% <concept_desc>Social and professional topics~History of programming languages</concept_desc>
% <concept_significance>300</concept_significance>
% </concept>
% </ccs2012>
% \end{CCSXML}
% 
% \ccsdesc[500]{Software and its engineering~General programming languages}
% \ccsdesc[300]{Social and professional topics~History of programming languages}
%% End of generated code

%% Keywords
%% comma separated list
\keywords{\todo}

%% \maketitle
%% Note: \maketitle command must come after title commands, author
%% commands, abstract environment, Computing Classification System
%% environment and commands, and keywords command.
\maketitle

%% body of the paper
\section{Introduction}\label{sec:intro}

The \textit{conformance testing} of programming language implementations is
essential to provide correct and consistent implementations of the
language semantics. Many programming languages have multiple implementations
rather than a single reference implementation. For example, Java uses a Java Virtual
Machine (JVM) to compile Java programs into JVM bytecode and execute them.
Developers are free to choose one of the existing JVM
implementations, such as OpenJ9, GraalVM, HotSpot, Zulu, and Corretto.
Python has the reference interpreter, CPython, in addition to diverse
interpreters, including PyPy, Jython, and IronPython.
Therefore, ensuring correct and consistent execution environments in different
implementations of the same language becomes crucial.
However, since manually maintaining conformance test suites for
real-world programming languages is cumbersome and labor-intensive, only a
small number of programming languages, such as JavaScript~\cite{test262}
and XML~\cite{xml-test-suite}, provide their official conformance test suites.
Thus, researchers have presented ways to test the
conformance of multiple implementations using differential
testing~\cite{diff-test} for compilers~\cite{csmith, deep-smith, diff-cpp-front,
diff-test-embedded}, interpreters~\cite{jit-picking, comfort}, virtual
machines~\cite{java-diff-test}, and debuggers~\cite{diff-debugger}.
To make differential testing for language implementations effective,
researchers have proposed various techniques to synthesize diverse programs,
such as generation-based fuzzing~\cite{csmith, jit-picking, diff-test-embedded, diff-debugger},
mutation-based fuzzing~\cite{java-diff-test, diff-cpp-front},
and deep learning~\cite{comfort, deep-smith}.

%----------------------------------------%

\textit{Graph coverage}~\cite{cov-def} is one of the most widely-used coverage criteria
in evaluating the quality of conformance tests.
Higher coverage of a conformance test suite denotes that it covers more
test requirements (TRs) of a given coverage criterion for language implementations.
Graph coverage helps generate tests that reach uncovered parts of software;
coverage-guided fuzzing (CGF)~\cite{afl} improves mutation-based fuzzing
by selecting mutation target programs using coverage information.
It also helps avoid an excessive number of conformance tests;
researchers have presented various test minimization techniques~\cite{test-minimize-survey}
to reduce the number of tests, and
\citet{cov-test-minimize} present coverage-guided test minimization.

%----------------------------------------%

One approach to making high-quality conformance tests is to use graph coverage
to generate tests for \textit{mechanized language specifications}.
%
While we can use code coverage rather than graph coverage
to generate tests for ``actual language implementations,''
it will lead to different coverage information for different implementations.
On the contrary, graph coverage for mechanized language specifications
will lead to uniform coverage information for multiple implementations.
%
Various programming languages, such as OCaml~\cite{ocaml-light-spec},
C~\cite{c-light-spec}, C++\cite{cpp-spec}, Java~\cite{k-java},
JavaScript~\cite{jiset}, and POSIX shell~\cite{posix-shell-spec},
have mechanized specifications that formally describe their
semantics using diverse metalanguages and frameworks, such as Ott~\cite{ott}, Skel~\cite{skel}, and the
$\kframework$ framework~\cite{kframework}.
%
Since mechanized specifications use functions to describe the semantics of language features,
we can easily convert them as directed graphs and traditional graph coverage
criteria for software work as they are.
For example, $\kjava$~\cite{k-java} is a mechanized specification for Java
defined with the $\kframework$ framework, which describes language semantics
as a set of reduction rules.
Consider a directed graph whose nodes are reduction rules and edges are
their dependencies in $\kjava$.
Then, we can measure the coverage of a test suite in the directed graph denoting $\kjava$
based on whether each test covers the test requirements of a graph coverage criterion.

%----------------------------------------%

\paragraph{\textbf{Challenges}}
However, graph coverage may not produce high-quality conformance tests
for mechanized language specifications.
Mechanized specifications are usually written in a modular way with helper functions.
Such a modular definition has the advantages of preventing duplicated or similar
definitions of language semantics, reducing the size of a mechanized
specification, and enhancing its readability.
At the same time, reusing the same helper function for different parts
may degrade the quality of conformance testing.

%----------------------------------------%

First, traditional graph coverage may not distinguish test requirements of
different language features when their semantics descriptions
use the same functions, degrading conformance testing quality.
For example, consider a mechanized specification for JavaScript that represents the
abstract algorithms described in the official language specification, ECMA-262~\cite{es13}.
Here, most of the semantics for the addition and subtraction operators are
defined using the same \textbf{EvaluateStringOrNumericBinaryExpression} algorithm as a helper function.
If conformance tests for the addition operator already cover the test
requirements in the algorithm, most conformance tests for the subtraction operator
are removed after the coverage-guided test minimization process.
However, real-world JavaScript engines are highly optimized and often have
specialized execution paths for different language features,
even when their semantics descriptions use the same functions.
Therefore, we need to test possible edge cases for the subtraction operator as
well, even though similar edge cases for the addition operator are already tested.

%----------------------------------------%

Furthermore, it may not distinguish test requirements of different
parts of the same language feature when their semantics descriptions
use the same functions, degrading the quality of conformance testing.
For example, consider the mechanized specification for JavaScript again.
In JavaScript, the \jscode{String.prototype.normalize} built-in API normalizes
a given string into a normalization form named by a given argument.
The definition of the semantics for this built-in API feature uses
the \textbf{ToString} algorithm as a helper function twice to represent
conversions to strings for 1) \jscode{this} value and 2) the first argument of the API call.
Assume that a conformance test suite already covers the test requirements in the
\textbf{ToString} algorithm thanks to various values for \jscode{this} value.
Then, there is no chance to generate new conformance tests that check edge cases
of the conversion from the first argument to string when performing coverage-guided fuzzing.

%----------------------------------------%

\paragraph{\textbf{This Work}}

To alleviate this problem, we introduce \textit{feature-sensitive (FS) coverage},
a novel coverage criterion to generate high-quality conformance tests for
programming language implementations. It is a general extension of graph coverage,
refining test requirements using the innermost enclosing language features.
FS coverage resolves the problem of sharing the same helper functions
for the semantics of different language features.
We also present a \textit{feature-call-path-sensitive (FCPS) coverage},
a variant of FS coverage with feature-call-paths from language features to test requirements.
FCPS coverage resolves the problem of sharing the same helper functions
for the semantics of different parts of the same language feature.
In addition, we extend both coverage criteria using the $k$-limiting approach as $k$-FS
coverage and $k$-FCPS coverage.
To evaluate the effectiveness of the new coverage criteria,
we apply them to a real-world programming language, JavaScript.
We select JavaScript as the evaluation target language because
1) it has the most up-to-date mechanized specification and
2) it has the official conformance test suite, Test262~\cite{test262}.
We extend $\jest$~\cite{jest}, the state-of-the-art JavaScript conformance test
synthesizer using coverage-guided mutational fuzzing, with various FS
and FCPS coverage criteria.
For the latest language specification (ES13, 2022), our tool automatically
synthesizes \inred{95,000} conformance tests in \inred{50} hours with five coverage criteria.
We evaluated the conformance of eight mainstream JavaScript implementations
(four engines and four transpilers) with the synthesized conformance tests
and discovered bugs in all of them.
The tool detected \inred{115} distinct conformance bugs (\inred{40} in engines
and \inred{75} in transpilers), \inred{80} of which were confirmed by
the developers and \inred{40} of which were newly discovered bugs.

%----------------------------------------%

\paragraph{\textbf{Contributions}}
We summarize our contributions as follows:
\begin{itemize}
  \item
    We introduce novel \textit{feature-sensitive (FS) coverage} to discriminate
    test requirements with the innermost enclosing language features to enhance
    the quality of conformance testing for programming language implementations.
    It can resolve the problem of sharing the same helper
    functions for the semantics of different language features.
  \item 
    We also present \textit{feature-call-path-sensitive (FCPS) coverage} as its
    variant with feature-call-paths from language features to test requirements
    to distinguish different parts in the semantics of the same language feature.
  \item
    We experimentally show that the new coverage criteria outperform
    the traditional coverage criteria in the
    context of conformance bug detection in eight mainstream
    JavaScript implementations (four engines and four transpilers) with the latest ECMA-262
    (ES13, 2022). The tool uncovered \inred{40} brand-new bugs.
\end{itemize}

\section{Related Work}\label{sec:related}

\begin{itemize}
  \item JavaScript language specificaiton: ECMA-262 (ES13, 2022)~\cite{es13}
  \item JavaScript tests: Test262~\cite{test262}
  \item JavaScript tools: engines~\cite{v8, jscore, graaljs, chakra,
    spidermonkey, xs}, static analyzers~\cite{safe, safe2, tajs, wala, jsai},
    debugger~\cite{jsexplain}, verification tools~\cite{javert, javert2,
    ad-safety, javanni}, symbolic execution~\cite{symbolic-js, sym-js, expo-se},
    concolic testing~\cite{jalangi, type-conc-test}.
  \item Domain Specific Language (DSL)~\cite{dsl-survey, dsl-survey2}
  \item JISET family: $\jiset$~\cite{jiset}, $\jest$~\cite{jest},
    $\jstar$~\cite{jstar}, $\jsaver$~\cite{jsaver}.
  \item Partial Evaluation~\cite{peval, peval-survey}, Program
    Transformation~\cite{trans-ai}.
  \item JavaScript Engine Fuzzer~\cite{montage, langfuzz, die, favocado,
    codealchemist, sofi, comfort, superion, fuzzilli, jsfunfuzz,
  ifuzzer}.
\end{itemize}

\todo

\section{Conclusion}\label{sec:conclusion}
To support correct and consistent implementations of programming language semantics,
conformance testing using graph coverage has been one of the most
promising approaches. However, because language implementations often utilize 
different execution paths even for the same functionalities,
traditional graph coverage does not produce high-quality conformance tests.
In this paper, we present novel coverage criteria especially designed
for language implementations: \textit{feature-sensitive (FS) coverage} and
\textit{feature-call-path-sensitive (FCPS) coverage}
by refining conventional test requirements using
enclosing language features and call paths.
We also extend both coverage criteria using the $k$-limiting approach as
$k$-FS coverage and $k$-FCPS coverage.
Our experiments show that the new coverage criteria outperform the
traditional coverage criteria in the context of conformance bug detection in
real-world JavaScript implementations.
We detected \inred{115} distinct conformance bugs (\inred{40} in engines
and \inred{75} in transpilers), \inred{80} of which were confirmed by the
developers and \inred{40} of which were newly discovered bugs.


% TODO in camera-ready
%% Acknowledgments
% \begin{acks}
% \end{acks}

%% Bibliography
\bibliography{ref}

\end{document}
