\section{Feature-Sensitive Coverages}\label{sec:fscov}

This section first formulates the general definition of graph coverages for a
given control-flow graph (CFG) and explains representative coverage metrics as
examples.
%
Then, we introduce a \textit{feature-sensitive (FS) coverages} as their
extensions to fully discriminate semantics between different language features.
%
Finally, we define \textit{feature-call-path-sensitive (FCPS) coverages} as its
variants to discriminate different parts in the semantics of the same language
features.




\subsection{Graph Coverages}

We formulate graph coverages by referring to its well-known
definition~\cite{testing}.




\subsubsection{Notations}
%
First, we define notations used in the definition of graph coverages.
%
A \textit{control-flow graph} (CFG) $\graph = (\nodeset, \inodeset,
\fnodeset, \edgeset)$ consists of:
\begin{enumerate}
  \item a set of \textit{nodes} $\nodeset$
  \item a set of \textit{initial nodes} $\inodeset \subseteq \nodeset$
  \item a set of \textit{final nodes} $\fnodeset \subseteq \nodeset$
  \item a set of \textit{edges} $\edgeset \subseteq \nodeset \times \nodeset$
\end{enumerate}
%
The notation $\node \rightarrow \node'$ denotes an edge from $\node$ to
$\node'$.
%
If an edge has a specific annotation, such as \name{\#t}, we use the notation
$\node \tedge \node'$.
%
In a given CFG, a \textit{path} $\pat \in \patset$ is a sequence of nodes, where
each pair of adjacent nodes is an edge as follows:
\[
  \patset = \{
    [\node_0, \cdots, \node_k] \mid
    \forall i < k, (\node_i, \node_{i+1}) \in \edgeset \wedge
    \node_k \in \fnodeset
  \}
\]
%
The notation $\node_0 \rightarrow \cdots \rightarrow \node_k$ denotes a path.
%
A path $\pat$ is a \textit{subpath} of another path $\pat'$ when 
%
A \textit{test path} is a path that starts at an initial node and ends at a
final node (i.e., $\node_0 \in \inodeset \wedge \node_k \in \fnodeset$).
%
Then, $\patmap{\graph} : \testset \rightarrow \patset$ is a mapping from a test
case $\test \in \testset$ to the corresponding test path in the graph $\graph$.

%----------------------------------------%

For example, consider the CFG $\graph$ depicted in Figure~\ref{fig:spec-cfg} and
a JavaScript program $\jscode{1 + 2n;}$ as a test case $\test$.
%
Then, the $\patmap{\graph}(\test)$ is the following test path:
\[
  \small
  \begin{array}{l}
    \cdots
    \call \overset{\text{{\bf Evaluation} of \esnt{AdditiveExpression}
      \esconst{+} \esnt{MultiplicativeExpression}}}{\rcolorbox{gray3}{\(
    1
    \call
    \overset{\textbf{EvaluateStringOrNumericBinaryExpression}}{\rcolorbox{gray2}{\(
    7 \flowsto \cdots \flowsto 8
    \call \overset{\textbf{ApplyStringOrNumericBinaryOperator}}{\colorbox{gray1}{\(
    11 \flowsto \cdots \flowsto 12
    \call \overset{\textbf{ToNumeric}}{\colorbox{white}{\(
    19 \flowsto \cdots \flowsto 20 \fedge \cdots \flowsto 22
    \)}}
    \ret 13 \fedge 14
    \)}}
    \)}}
    \)}}

    \vspace*{1em}\\

    \lcolorbox{gray3}{\(
    \lcolorbox{gray2}{\(
    \colorbox{gray1}{\(
    \call \overset{\textbf{ToNumeric}}{\colorbox{white}{\(
    19 \flowsto \cdots \flowsto 20 \tedge \colorbox{lightred}{21} \flowsto 22
    \)}}
    \ret 15 \fedge 16 \tedge \colorbox{lightred}{17} \flowsto 18
    \)}
    \ret 9 \tedge 10
    \)}
    \ret 2 \tedge 3
    \)}
    \ret \cdots\\
  \end{array}
\]





% \subsubsection{Graph Coverage}
% 
% \todo
% 
% % Now, we could define a \textit{graph coverage} with the corresponding
% % 1) \textit{test requirements} and 2) \textit{}:
% % \begin{definition}
% % For example, a JavaScript program \jscode{1 + 2;} is given 
% % When a testcase is given
