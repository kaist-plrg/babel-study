\section{Introduction}\label{sec:intro}

Developers utilize JavaScript engines in a wide range of fields.
%
While JavaScript was a simple scripting language for client-side programming in
web browsers, it has now become one of the most popular programming languages.
%
It happens because diverse host environments support JavaScript engines for
server-side programming, cross-platform desktop or mobile applications, PDF
files, and even embedded systems.
%
For a consistent execution environment anywhere, JavaScript engines must conform
to ECMA-262~\cite{es13}, the official ECMAScript language specification.
%
It defines JavaScript syntax in a variant of the extended Backus–Naur form
(EBNF) and the language semantics using abstract algorithms consisting of
well-organized English sentences as steps.
%
To check the conformance between the specification and JavaScript engines,
developers usually utilized Test262~\cite{test262}, the official
\textit{conformance test} suite.
%
It consists of JavaScript programs with assertions that check the expected
behaviors according to the language semantics described in the specification.
%
Unfortunately, Test262 is manually maintained without any automated check for
the correctness of the test programs.
%
Hence, it is labor-intensive to retain them as strictly conforming to the
specification.

%----------------------------------------%

It is challenging to automatically synthesize correct JavaScript conformance
tests.
%
First, JavaScript has complex language semantics with a highly dynamic nature,
such as first-class functions, first-class property names, implicit control
flows (e.g., getters/setters), asynchronous function calls (\code{Promise}),
mutable prototypes, and even dynamic code execution (\code{eval}).
%
Such complex semantics make it difficult to cover all possible edge cases in the
semantics of various language features.
%
Second, JavaScript is fast-evolving with new language features, while the
language specification is written in a natural language.
%
Since 2015, the language specification is annually released to quickly adapt
users' demands to the language.
%
Hence, there have already been eight major updates of ECMA-262, and the latest
version (ES13, 2022) is 846 pages.

%----------------------------------------%

Therefore, most prior work~\cite{montage, langfuzz, die, favocado,
codealchemist, sofi} only focuses on the detection of crashing bugs or
vulnerabilities rather than conformance bugs in JavaScript engines.
%
Existing techniques utilize fuzz testing (or fuzzing), widely used in compiler
bug detection by randomly generating or mutating test inputs.
%
Even though they show outstanding results in detecting crash bugs or
vulnerabilities in JavaScript engines, they almost neglect the detection of
conformance bugs.
%
On the other hand, $\comfort$~\cite{comfort} targets conformance bugs by
leveraging a deep-learning-based program synthesis~\cite{deep-smith} with
JavaScript values used as function arguments collected from the language
specification.
%
However, it utilizes ECMA-262 only for program synthesis but not for conformance
checks.
%
Thus, it heavily relies on the results of differential testing with multiple
engines and cannot synthesize conformance tests strictly conforming to the
specification.

%----------------------------------------%

\paragraph{\textbf{Challenges}}
%
Unlike existing fuzzers on JavaScript engines, $\jest$~\cite{jest} automatically
synthesizes conformance tests directly from a given version of ECMA-262 without
any engines.
%
It first synthesizes JavaScript programs using coverage-guided
fuzzing~\cite{afl} with \textit{statement and branch coverages} in the language
specification rather than engines.
%
Then, it automatically injects assertions into the synthesized programs based on
their expected final states in the language specification.
%
While it has a remarkable result in detecting conformance bugs, it still has two
limitations: 1) \textit{removal of meaningful test programs} and 2) \textit{no
consideration of transpilers}.

%----------------------------------------%

First, statement and branch coverages might \textit{remove meaningful test
programs} in the final test suite due to the lack of ability to discriminate
similar edge cases in different language features.
%
JavaScript \textit{language features} are either 1) syntactic features or 2)
built-in APIs.
%
In the language specification, the semantics of different language features
might be defined using the same abstract algorithms.
%
Even different semantics parts of the same features might be defined using the
same algorithms.
%
Unfortunately, the program synthesizer with statement and branch coverages
cannot recognize whether different semantics parts are touched in above cases
because such coverage metrics are context-insensitive.
%
As a result, it lowers the quality of the fuzzing guidance and might even remove
meaningful test programs in the final test suite.
%
In addition, they do not have the ability to discriminate various combinations
between features.
%
We describe more details of these problems with simple examples in
Section~\ref{sec:motivation}.

%----------------------------------------%

Second, prior work focuses on conformance checks only for engines but does not
consider \textit{transpilers}.
%
However, nowadays, JavaScript transpilers have become essential tools in the
deployment process of JavaScript applications for various purposes.
%
For example, Babel and SWC desugar new language features into
backward-compatible versions of the language.
%
The code minifiers, such as \code{terser}, compress code size to reduce the
network communication cost, and JavaScript Obfuscator obfuscates code to make it
hard to understand and reverse-engineering.
%
Therefore, ensuring the conformance of transpilers becomes crucial more and
more.


%----------------------------------------%

\paragraph{\textbf{This Work}}

This paper introduces a novel \textit{feature-sensitive coverage} as a general
extension of any coverage metrics.  
%
First, we formally define a \textit{sensitive coverage} parametric with a given
set of \textit{views} $\viewset$ as an abstraction of enclosing calling contexts.
%
Then, we represent the feature-sensitive coverage as an instance of the sensitive
coverage with \textit{feature-sensitive views}, which consists of two elements:
%
1) \textit{language features} to discriminate semantics parts of different
features and 2) \textit{intra-feature call graphs} for the different semantics
parts of the same features.
%
Moreover, we extend it to cover pair-wise combinations between different
language features.
%
Then, we generalize prior work to synthesize conformance tests for both
JavaScript engines and transpilers.
%
To actualize our techniques, we implemented $\tool$ by extending a
state-of-the-art JavaScript conformance test synthesizer, $\jest$.
%
We applied our tool to the latest language specification (ES13, 2022) for
evaluation, and it automatically synthesized \inred{5,000} conformance tests
in \inred{100} hours.
%
We tested \inred{eight} mainstream targets (\inred{four} engines and
\inred{four} transpilers) with synthesized conformance tests and discovered
bugs in all of them.
%
Our tool detected \inred{50} unique conformance bugs (\inred{20} in engines
and \inred{30} in transpilers), while the baseline tool detected only
\inred{16} engine bugs.
%
We had reported all detected bugs, developers confirmed all of them, and
\inred{40} were undiscovered bugs.
%
\inred{Furthermore, 25 test cases detected by our tool were added to Test262,
the official JavaScript conformance test suite.}

%----------------------------------------%

\paragraph{\textbf{Contributions}}
%
We summarize our contributions as follows:
%
\begin{itemize}

  \item
    We introduce a novel \textit{feature-sensitive coverage} as a general
    extension of any coverage metrics to discriminate covered elements based on
    enclosing calling contexts that consist of \textit{language features} and
    \textit{intra-feature call graphs}.
    %
    We also define its extension to cover pair-wise combinations between
    different language features.

  \item
    We extended $\jest$ to $\tool$ with the feature-sensitive coverage and
    the synthesis of conformance tests for transpilers.
    %
    Note that it is the first tool that automatically synthesizes conformance
    tests for both JavaScript engines and transpilers.

  \item
    We experimentally show that our tool outperforms the baseline tool in the
    context of conformance bug detection in \inred{eight} mainstream targets
    (\inred{four} engines and \inred{four} transpilers) with the latest ECMA-262
    (ES13, 2022).
    %
    As a result, $\tool$ detected \inred{50} unique conformance bugs (\inred{20}
    in engines and \inred{30} in transpilers), while the baseline tool detected
    only \inred{16} engine bugs.
\end{itemize}
