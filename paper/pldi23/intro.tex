\section{Introduction}\label{sec:intro}

The \textit{conformance testing} of programming language implementations is
essential to provide correct and consistent implementations of the
language semantics. Many programming languages have multiple implementations
rather than a single reference implementation. For example, Java uses a Java Virtual
Machine (JVM) to compile Java programs into JVM bytecode and execute them.
Developers are free to choose one of the existing JVM
implementations, such as OpenJ9, GraalVM, HotSpot, Zulu, and Corretto.
Python has the reference interpreter, CPython, in addition to diverse
interpreters, including PyPy, Jython, and IronPython.
Therefore, ensuring correct and consistent execution environments in different
implementations of the same language becomes crucial.
However, since manually maintaining conformance test suites for
real-world programming languages is cumbersome and labor-intensive, only a
small number of programming languages, such as JavaScript~\cite{test262}
and XML~\cite{xml-test-suite}, provide their official conformance test suites.
Thus, researchers have presented ways to test the
conformance of multiple implementations using differential
testing~\cite{diff-test} for compilers~\cite{csmith, deep-smith, diff-cpp-front,
diff-test-embedded}, interpreters~\cite{jit-picking, comfort}, virtual
machines~\cite{java-diff-test}, and debuggers~\cite{diff-debugger}.
To make differential testing for language implementations effective,
researchers have proposed various techniques to synthesize diverse programs,
such as generation-based fuzzing~\cite{csmith, jit-picking, diff-test-embedded, diff-debugger},
mutation-based fuzzing~\cite{java-diff-test, diff-cpp-front},
and deep learning~\cite{comfort, deep-smith}.

%----------------------------------------%

\textit{Graph coverage}~\cite{cov-def} is one of the most widely-used coverage criteria
in evaluating the quality of conformance tests.
Higher coverage of a conformance test suite denotes that it covers more
test requirements (TRs) of a given coverage criterion for language implementations.
Graph coverage helps generate tests that reach uncovered parts of software;
coverage-guided fuzzing (CGF)~\cite{afl} improves mutation-based fuzzing
by selecting mutation target programs using coverage information.
It also helps avoid an excessive number of conformance tests;
researchers have presented various test minimization techniques~\cite{test-minimize-survey}
to reduce the number of tests, and
\citet{cov-test-minimize} present coverage-guided test minimization.

%----------------------------------------%

One approach to making high-quality conformance tests is to use graph coverage
to generate tests for \textit{mechanized language specifications}.
%
While we can use code coverage rather than graph coverage
to generate tests for ``actual language implementations,''
it will lead to different coverage information for different implementations.
On the contrary, graph coverage for mechanized language specifications
will lead to uniform coverage information for multiple implementations.
%
Various programming languages, such as OCaml~\cite{ocaml-light-spec},
C~\cite{c-light-spec}, C++\cite{cpp-spec}, Java~\cite{k-java},
JavaScript~\cite{jiset}, and POSIX shell~\cite{posix-shell-spec},
have mechanized specifications that formally describe their
semantics using diverse metalanguages and frameworks, such as Ott~\cite{ott}, Skel~\cite{skel}, and the
$\kframework$ framework~\cite{kframework}.
%
Since mechanized specifications use functions to describe the semantics of language features,
we can easily convert them as directed graphs and traditional graph coverage
criteria for software work as they are.
For example, $\kjava$~\cite{k-java} is a mechanized specification for Java
defined with the $\kframework$ framework, which describes language semantics
as a set of reduction rules.
Consider a directed graph whose nodes are reduction rules and edges are
their dependencies in $\kjava$.
Then, we can measure the coverage of a test suite in the directed graph denoting $\kjava$
based on whether each test covers the test requirements of a graph coverage criterion.

%----------------------------------------%

\paragraph{\textbf{Challenges}}
However, graph coverage may not produce high-quality conformance tests
for mechanized language specifications.
Mechanized specifications are usually written in a modular way with helper functions.
Such a modular definition has the advantages of preventing duplicated or similar
definitions of language semantics, reducing the size of a mechanized
specification, and enhancing its readability.
At the same time, reusing the same helper function for different parts
may degrade the quality of conformance testing.

%----------------------------------------%

First, traditional graph coverage may not distinguish test requirements of
different language features when their semantics descriptions
use the same functions, degrading conformance testing quality.
For example, consider a mechanized specification for JavaScript that represents the
abstract algorithms described in the official language specification, ECMA-262~\cite{es13}.
Here, most of the semantics for the addition and subtraction operators are
defined using the same \textbf{EvaluateStringOrNumericBinaryExpression} algorithm as a helper function.
If conformance tests for the addition operator already cover the test
requirements in the algorithm, most conformance tests for the subtraction operator
are removed after the coverage-guided test minimization process.
However, real-world JavaScript engines are highly optimized and often have
specialized execution paths for different language features,
even when their semantics descriptions use the same functions.
Therefore, we need to test possible edge cases for the subtraction operator as
well, even though similar edge cases for the addition operator are already tested.

%----------------------------------------%

Furthermore, it may not distinguish test requirements of different
parts of the same language feature when their semantics descriptions
use the same functions, degrading the quality of conformance testing.
For example, consider the mechanized specification for JavaScript again.
In JavaScript, the \jscode{String.prototype.normalize} built-in API normalizes
a given string into a normalization form named by a given argument.
The definition of the semantics for this built-in API feature uses
the \textbf{ToString} algorithm as a helper function twice to represent
conversions to strings for 1) \jscode{this} value and 2) the first argument of the API call.
Assume that a conformance test suite already covers the test requirements in the
\textbf{ToString} algorithm thanks to various values for \jscode{this} value.
Then, there is no chance to generate new conformance tests that check edge cases
of the conversion from the first argument to string when performing coverage-guided fuzzing.

%----------------------------------------%

\paragraph{\textbf{This Work}}

To alleviate this problem, we introduce \textit{feature-sensitive (FS) coverage},
a novel coverage criterion to generate high-quality conformance tests for
programming language implementations. It is a general extension of graph coverage,
refining test requirements using the innermost enclosing language features.
FS coverage resolves the problem of sharing the same helper functions
for the semantics of different language features.
We also present a \textit{feature-call-path-sensitive (FCPS) coverage},
a variant of FS coverage with feature-call-paths from language features to test requirements.
FCPS coverage resolves the problem of sharing the same helper functions
for the semantics of different parts of the same language feature.
In addition, we extend both coverage criteria using the $k$-limiting approach as $k$-FS
coverage and $k$-FCPS coverage.
To evaluate the effectiveness of the new coverage criteria,
we apply them to a real-world programming language, JavaScript.
We select JavaScript as the evaluation target language because
1) it has the most up-to-date mechanized specification and
2) it has the official conformance test suite, Test262~\cite{test262}.
We extend $\jest$~\cite{jest}, the state-of-the-art JavaScript conformance test
synthesizer using coverage-guided mutational fuzzing, with various FS
and FCPS coverage criteria.
For the latest language specification (ES13, 2022), our tool automatically
synthesizes \inred{95,000} conformance tests in \inred{50} hours with five coverage criteria.
We evaluated the conformance of eight mainstream JavaScript implementations
(four engines and four transpilers) with the synthesized conformance tests
and discovered bugs in all of them.
The tool detected \inred{115} distinct conformance bugs (\inred{40} in engines
and \inred{75} in transpilers), \inred{80} of which were confirmed by
the developers and \inred{40} of which were newly discovered bugs.

%----------------------------------------%

\paragraph{\textbf{Contributions}}
We summarize our contributions as follows:
\begin{itemize}
  \item
    We introduce novel \textit{feature-sensitive (FS) coverage} to discriminate
    test requirements with the innermost enclosing language features to enhance
    the quality of conformance testing for programming language implementations.
    It can resolve the problem of sharing the same helper
    functions for the semantics of different language features.
  \item 
    We also present \textit{feature-call-path-sensitive (FCPS) coverage} as its
    variant with feature-call-paths from language features to test requirements
    to distinguish different parts in the semantics of the same language feature.
  \item
    We experimentally show that the new coverage criteria outperform
    the traditional coverage criteria in the
    context of conformance bug detection in eight mainstream
    JavaScript implementations (four engines and four transpilers) with the latest ECMA-262
    (ES13, 2022). The tool uncovered \inred{40} brand-new bugs.
\end{itemize}
