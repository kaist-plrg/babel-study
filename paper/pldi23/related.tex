\section{Related Work}\label{sec:related}

\paragraph{\textbf{Coverage Criteria in Software Testing}}
Coverage criteria in software testing are essential in measuring the quality of test inputs.
The most common coverage criteria are structural coverages in a given program's
control-flow graph (CFG)~\cite{cov-mcdc,cov-def} also utilizing
data-flow information~\cite{cov-dataflow} and type information~\cite{cov-partition}.
On the other hand, model-based coverage~\cite{cov-model-book} criteria consider
specialized abstract behavior models and define the test requirements in the model.
Such models include state transitions~\cite{cov-model-api},
autonomous driving systems~\cite{cov-param},
deep neural networks (DLLs)~\cite{cov-neuron,cov-k-neuron,cov-concolic-dll,
cov-fuzz-tensor, cov-fuzz-dl}.
However, there are no specialized coverage criteria for programming language
tools, such as compilers, interpreters, and transpilers.
In this paper, we first presented feature-sensitive coverages as general
extensions of graph coverages for programming language tools to discriminate
test requirements based on enclosing language features or feature-call-paths.

%----------------------------------------%

\paragraph{\textbf{Mechanized Specification}}

Researchers have presented mechanized specifications to formally describe the
semantics of diverse programming languages, such as
OCaml~\cite{ocaml-light-spec}, C~\cite{c-light-spec}, C++\cite{cpp-spec},
Java~\cite{k-java}, and POSIX shell~\cite{posix-shell-spec}.
%
At the same time, general metalanguages or frameworks for mechanized language
specifications have also emerge as.
%
For example, \citet{ott} presented Ott as a tool that compiles language
semantics into proof assistant code for Coq, HOL, and Isabelle/HOL and supports
a metalanguage used in defining language semantics as inference rules.
%
The $\kframework$ framework~\cite{kframework} proposed a formalism for writing
operational semantics and provides a derivation of verifiers directly from the
semantics.
%
\citet{skel} developed a skeletal semantics framework in Coq for creating
big-step semantics by focusing on the structure of the semantics.

%----------------------------------------%

For the JavaScript programming language, diverse mechanized specifications have
been presented based on ECMA-262~\cite{es13}, the official language
specification written in a natural language.
%
For example, $\lambdajs$~\cite{lambdajs} was the first core calculus of ES5.1
and provided a desugaring of non-core language features.
%
KJS~\cite{kjs} utilizes the $\kframework$ framework, and \citet{javert}
presented a metalanguage, $\jsil$, for ES5.1.
%
Researchers have used such mechanized specification in diverse fields:
verification~\cite{javert}, symbolic execution~\cite{javert2}, abstract
interpretation-based static analysis~\cite{wala, tajs, jsai, safe, safe2}, and
double debugger~\cite{jsexplain}.
%
However, most JavaScript mechanized specifications focused on only ES5.1,
released in 2011, and required manual description of the semantics.
%
On the other hand, $\esmeta$ supports a metalanguage $\ires$ for the latest
version of ECMA-262 and the automatic extraction of mechanized specification
used in conformance test synthesis~\cite{jest}, specification type
analysis~\cite{jstar}, and static analyzer derivation~\cite{jsaver}.
%
Hence, we implemented $\tool$ based on $\esmeta$ to synthesize conformance tests
from the latest specification (ES13, 2022) with feature-sensitive coverages.

%----------------------------------------%

\paragraph{\textbf{Conformance Testing for JavaScript}}
%
Diverse host environments support JavaScript engines for server-side
programming, cross-platform desktop or mobile applications, PDF files, and even
embedded systems.
%
In addition, JavaScript transpilers become essential tools in the deployment
process of JavaScript applications.
%
Therefore, ensuring the conformance of engines and transpilers according to the
language specification is crucial to consistent execution environments for
JavaScript.
%
The current solution is to maintain conformance tests by hand, and engine and
transpiler developers commonly utilize Test262~\cite{test262}, the official
JavaScript conformance test suite.
%
Researchers have focused on testing JavaScript engines to detect bugs using
generation-based fuzzing~\cite{die, codealchemist, favocado, sofi} and
mutation-based fuzzing~\cite{die, codealchemist, ifuzzer, superion}.
%
In addition, they often utilize deep learning~\cite{montage, comfort} to
generate JavaScript programs in advance and differential
testing~\cite{jit-picking} to check the correctness of execution results.
%
However, most existing techniques focused on detecting crashing bugs or security
vulnerabilities rather than conformance bugs.
%
While $\comfort$~\cite{comfort} targets conformance bugs, it heavily relies on
the results of differential testing instead of the language semantics described
in ECMA-262.
%
On the other hand, $\jest$~\cite{jest} is the first tool that automatically
synthesizes JavaScript conformance tests according to the language semantics
described in the language specification.
%
We implemented $\tool$ by augmenting it with $k$-FS and $k$-FCPS coverages and
outperformed the ability of conformance bug detection of the $\jest$.
