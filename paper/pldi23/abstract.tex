\begin{abstract}
  A \textit{conformance test} suite is essential to support consistent execution
  environments for JavaScript.
  %
  It is challenging to automatically synthesize them because of 1) the complex
  language semantics with a highly dynamic nature and 2) the fast-evolving
  language specification written in a natural language, English.
  %
  Researchers have presented a way to automatically synthesize conformance tests
  for JavaScript using \textit{coverage-guided fuzzing} with the control-flow
  graph (CFG) in the language specification.
  %
  However, existing work utilizes simple \textit{node/branch coverages}.
  %
  We found that their test requirements are not enough to distinguish the
  semantics of different \textit{language features} defined with shared abstract
  algorithms in the specification.
  %
  As a result, it weakens the fuzzing guidance and even removes meaningful
  conformance tests in the final test suite.
  %
  In addition, existing work focuses on conformance checks only for engines but
  does not consider transpilers even though developers heavily utilize them.

  %----------------------------------------%

  This paper introduces a novel \textit{feature-sensitive coverage}, which
  discriminates the test requirements with their enclosing language features.
  %
  We observe that specific abstract algorithms in the JavaScript language
  specification are coupled with language features.
  %
  We utilize this information to enhance the quality of the coverage-guided
  fuzzing for conformance test synthesis.
  %
  fuzzing for conformance test synthesis.
  %
  We implemented $\tool$ by extending a state-of-the-art JavaScript conformance
  test synthesizer, $\jest$, with our feature-sensitive coverage.
  %
  For the latest language specification (ES13, 2022), our tool automatically
  synthesized \inred{5,000} conformance tests in \inred{100} hours.
  %
  We checked conformance of not only engines but also transpilers with the
  synthesized conformance tests for evaluation.
  %
  The evaluation targets were \inred{eight} mainstream tools (\inred{four}
  engines and \inred{four} transpilers), and we discovered bugs in all of them.
  %
  Our tool detected \inred{50} unique conformance bugs (\inred{20} in engines
  and \inred{30} in transpilers), while the baseline tool detected only
  \inred{16} engine bugs.
  %
  We had reported all detected bugs, developers confirmed all of them, and
  \inred{40} were newly discovered bugs.
\end{abstract}
